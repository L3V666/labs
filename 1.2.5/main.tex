\documentclass[a4paper, 12pt]{article}

% Русский язык
\usepackage[T2A]{fontenc}
\usepackage[utf8]{inputenc}
\usepackage[english,russian]{babel}

% Картинки
\usepackage{graphicx}
\graphicspath{{images/}}
\DeclareGraphicsExtensions{.pdf,.png,.jpg}
\usepackage{wrapfig}

% Математика
\usepackage{amsmath,amsfonts,amssymb,amsthm,mathtools} 

% Параметры страницы
\usepackage[left=2cm,right=2cm,top=2cm,bottom=3cm,bindingoffset=0cm]{geometry}

\usepackage{enumitem,xparse}
\usepackage{caption}
\usepackage{subcaption}
\usepackage{float}

\begin{document}

\newgeometry{left=2cm,right=2cm,top=2cm,bottom=1cm,bindingoffset=0cm}
\begin{titlepage}
    
    \begin{center}
        \vspace*{5cm}
        \Huge Московский Физико-Технический Институт
        \vspace*{2cm}\\
        \LARGE Отчет по эксперименту
        \\\vspace*{0.25cm}
        
        \noindent\rule{\textwidth}{1pt}
        \vspace*{-0.25cm}
        
        \huge \textbf{1.2.5 \\ Исследование вынужденной регулярной прецессии гироскопа}
        \noindent\rule{\textwidth}{1pt}

        \vspace{6cm}
        \vfill
        \begin{flushright}
            \begin{minipage}{.4\textwidth}
            \Large Выполнил:\\ Студент 1 курса ФАКТ\\ Группа Б03-504 \\Подмосковнов Лев\\
            \end{minipage}
        \end{flushright}
        
        \vfill
        \normalsize Долгопрудный \\2025
        
    \end{center}
\end{titlepage}
\restoregeometry

\setcounter{page}{2}

\section*{Теоритические сведения}

\[\Omega=\frac{M}{I_z\omega_0\sin{\alpha}}=\frac{m_{\text{г}}gl_{\text{ц}}\sin{\alpha}}{I_z\omega_0\sin{\alpha}}=\frac{m_{\text{г}}gl_{\text{ц}}}{I_z\omega_0}\]

\[\Omega=\frac{mgl}{I_z\omega_0}\]

\[I_0=I_{\text{ц}}\frac{T_0^2}{T_{\text{ц}}^2}\]

\section*{Ход работы}

\begin{table}[h]
    \begin{minipage}{0.36 \textwidth}
        \centering
        \begin{tabular}{|c|c|c|c|}
        \hline
        $N$ & $t$, с & $\Omega$, $10^{-2}$c & $C$, $10^{-4}$c \\ 
        \hline
            1 & 136.2 & 4.61 & 15.38 \\ \hline
            1 & 136.7 & 4.59 & 15.32 \\ \hline
            1 & 136.4 & 4.60 & 15.35 \\ \hline
            1 & 136.5 & 4.60 & 15.34 \\ \hline
            1 & 136.5 & 4.60 & 15.34 \\ \hline
            \hline
        \end{tabular}
        \caption*{$m=76$ г, \\ $M=9.02\cdot 10^{-2}$ Н*м}
    \end{minipage}
    \begin{minipage}{0.36 \textwidth}
        \centering
        \begin{tabular}{|c|c|c|c|}
          \hline
          $N$ & $t$, с & $\Omega$, $10^{-2}$c & $C$, $10^{-4}$c \\ 
            \hline
                2 & 220.0 & 5.71 & 9.52 \\ \hline
                2 & 221.0 & 5.68 & 9.48 \\ \hline
                2 & 221.5 & 5.67 & 9.46 \\ \hline
                2 & 220.0 & 5.71 & 9.52 \\ \hline
                2 & 220.0 & 5.71 & 9.52 \\ \hline
            \hline
        \end{tabular}
        \caption*{$m=93$ г, \\ $M=1.1\cdot 10^{-1}$ Н*м}
    \end{minipage}
    \begin{minipage}{0.36 \textwidth}
        \centering
        \begin{tabular}{|c|c|c|c|}
            \hline
            $N$ & $t$, с & $\Omega$, $10^{-2}$c & $C$, $10^{-4}$c \\ 
            \hline
                3 & 266.4 & 7.07 & 7.86 \\ \hline
                3 & 266.3 & 7.07 & 7.86 \\ \hline
                3 & 266.5 & 7.07 & 7.86 \\ \hline
                3 & 266.2 & 7.08 & 7.87 \\ \hline
                3 & 266.3 & 7.07 & 7.86 \\ \hline
            \hline
        \end{tabular}
        \caption*{$m=116$ г, \\ $M=1.38\cdot 10^{-1}$ Н*м}
    \end{minipage}
    \begin{minipage}{0.36 \textwidth}
        \centering
        \begin{tabular}{|c|c|c|c|}
            \hline
            $N$ & $t$, с & $\Omega$, $10^{-2}$c & $C$, $10^{-4}$c \\ 
            \hline
            3 & 224.1 & 8.41 & 9.35 \\ \hline
            3 & 224.1 & 8.41 & 9.35 \\ \hline
            3 & 224.2 & 8.40 & 9.34 \\ \hline
            3 & 224.2 & 8.40 & 9.34 \\ \hline
            3 & 224.1 & 8.41 & 9.35 \\ \hline
            \hline
        \end{tabular}
        \caption*{$m=138$ г, \\ $M=1.64\cdot 10^{-1}$ Н*м}
    \end{minipage}
    \begin{minipage}{0.36 \textwidth}
        \centering
        \begin{tabular}{|c|c|c|c|}
            \hline
            $N$ & $t$, с & $\Omega$, $10^{-2}$c & $C$, $10^{-4}$c \\ 
            \hline
            4 & 237.1 & 10.59 & 8.83 \\ \hline
            4 & 237.2 & 10.59 & 8.83 \\ \hline
            4 & 237.3 & 10.59 & 8.83 \\ \hline
            4 & 237.2 & 10.59 & 8.83 \\ \hline
            4 & 237.2 & 10.59 & 8.83 \\ \hline
            \hline
        \end{tabular}
        \caption*{$m=174$ г, \\ $M=2.07\cdot 10^{-1}$ Н*м}
    \end{minipage}
\end{table}

\clearpage

\begin{figure}[H]
    \centering
    \includegraphics[width=0.75\textwidth]{1.png}
    \caption*{Зависимость $\Omega$ от $M$ и аппроксимация линейной функцией по МНК.}
\end{figure}

Момент инерции ротора равен: $I_0 = (0.86\pm0.01) \cdot 10^{-3}$ кг\cdot \text{м}^2

Погрешность измерения $\Omega$: $\sigma_{\Omega}=0.03$

Определить угловую ротора можно по формуле $\omega_0=\frac{1}{kI_0}$, где $k$ коэффициент наклона графика $\Omega(M)$. $\omega_0=2325$ c^{-1}. \text{Частоту можно найти по формуле} $\nu=\frac{\omega_0}{2\pi}=370$ Гц.

\subsection*{Вычисление момента инерции}

Для каждого груза момент силы трения будет разный:

$m=76$ г, $M = 3.07 \cdot 10^{-3}$ Н \cdot \text{м}

$m=93$ г, $M = 1.9 \cdot 10^{-3}$ Н \cdot \text{м}

$m=116$ г, $M = 1.57 \cdot 10^{-3}$ Н \cdot \text{м}

$m=138$ г, $M = 1.87 \cdot 10^{-3}$ Н \cdot \text{м}

$m=174$ г, $M = 1.77 \cdot 10^{-3}$ Н \cdot \text{м}

\subsection*{Измерение частоты вращения ротора с помощью фигуры Лиссажу}

\begin{figure}[H]
    \centering
    \includegraphics[width=0.75\textwidth]{2.png}
    \caption*{Фигура Лиссажу}
\end{figure}

Подобранная частота: $\nu=400\pm 0.5$ Гц

\section*{Вывод}

В данной работе были полученны частоты вращения ротора гироскопа двумя способами с разными погрешностями:
\begin{itemize}
    \item Через прецессию гироскопа: $\nu_{1} = 370 \text{ Гц}$
    \item Через фигуры Лиссажу:      $\nu_{2} = 400 \text{  Гц }$
\end{itemize}

Сравнивая результаты приходим к выводу что при расчете 1 способом на значения оказывает влияние погрешность в отличии от способа 2 где используются более точные приборы.

Также был оценен момент силы трения, действующий на ось гироскопа $ M \approx 10^{-3} \text{ } \text{Н} \cdot \text{м} $. Он оказался достаточно мал по сравнению с моментом силы тяжести груза, подвешенного на ось гироскопа, но достаточным для поворота гироскопа в сторону направления силы тяжести груза. Для его более точной оценки необходима более точная шкала определения отклонения гироскопа от начального уровня, которой, к сожалению, не было.

\end{document}