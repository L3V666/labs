\documentclass[a4paper, 12pt]{article}

% Русский язык
\usepackage[T2A]{fontenc}
\usepackage[utf8]{inputenc}
\usepackage[english,russian]{babel}

% Картинки
\usepackage{graphicx}
\graphicspath{{images/}}
\DeclareGraphicsExtensions{.pdf,.png,.jpg}
\usepackage{wrapfig}

% Математика
\usepackage{amsmath,amsfonts,amssymb,amsthm,mathtools} 

% Параметры страницы
\usepackage[left=2cm,right=2cm,top=2cm,bottom=3cm,bindingoffset=0cm]{geometry}

\usepackage{enumitem,xparse}
\usepackage{caption}
\usepackage{subcaption}
\usepackage{float}

\usepackage{hyperref}

\begin{document}

\newgeometry{left=2cm,right=2cm,top=2cm,bottom=1cm,bindingoffset=0cm}
\begin{titlepage}
    
    \begin{center}
        \vspace*{5cm}
        \Huge Московский Физико-Технический Институт
        \vspace*{2cm}\\
        \LARGE Отчет по эксперименту
        \\\vspace*{0.25cm}
        
        \noindent\rule{\textwidth}{1pt}
        \vspace*{-0.25cm}
        
        \huge \textbf{1.3.1 \\Определение модуля Юнга на основе исследования деформаций растяжения и изгиба}
        \noindent\rule{\textwidth}{1pt}


       \vfill
        \begin{flushright}
            \begin{minipage}{.4\textwidth}
            \Large Выполнил:\\ Студент 1 курса ФАКТ\\ Группа Б03-504 \\Подмосковнов Лев\\
            \end{minipage}
        \end{flushright}
        
        \vfill
        \normalsize Долгопрудный \\2025
        
    \end{center}
\end{titlepage}
\restoregeometry

\setcounter{page}{2}

\section*{Определение модуля Юнга по измерениям изгиба балки}

\includegraphics[scale=0.5]{1.jpg}

\subsection*{Теоритические сведения}

\[E=\frac{Pl^3}{4ab^3y_{max}}\]

\newpage

\subsection*{Результаты измерений и обработка данных}

Расстояние между призмами: $l=(50.5 \pm 0.5)$ см

\begin{table}[h]
    \begin{center}
        \begin{tabular}{|c|c|c|c|}
            \hline 
            & Дерево 1 & Дерево 2 & Сталь \\ \hline
            Ширина $a$, мм & 19.0 & 18.9 & 19.0 \\ \hline
            Высота $b$, мм & 10.2 & 10.1 & 3.9 \\ \hline
        \end{tabular}
    \end{center}
\end{table}

\begin{table}[h]
    \begin{minipage}{0.3 \textwidth}
        \centering
        \begin{tabular}{|c|c|}
            \hline
            $P$, Н & $y_{max}$, мм \\ \hline
            0.00 & 0.00 \\ \hline
            4.46 & 0.67 \\ \hline
            9.11 & 1.35 \\ \hline
            13.99 & 2.06  \\ \hline
            18.9 & 2.77 \\ \hline
            13.99 & 2.20 \\ \hline
            9.11 & 1.41 \\ \hline
            4.46 & 0.69 \\ \hline
            0.00 & 0.02 \\ \hline
        \end{tabular}
        \caption*{Дерево-1}
    \end{minipage}
    \hfil
    \begin{minipage}{0.3 \textwidth}
        \centering
        \begin{tabular}{|c|c|}
            \hline
            $P$, Н & $y_{max}$, мм \\ \hline
            0.00 & 0.00 \\ \hline
            4.46 & 0.65 \\ \hline
            9.11 & 1.35 \\ \hline
            13.99 & 2.07  \\ \hline
            18.9 & 2.81 \\ \hline
            13.99 & 2.11 \\ \hline
            9.11 & 1.43 \\ \hline
            4.46 & 0.74 \\ \hline
            0.00 & 0.03 \\ \hline
        \end{tabular}
        \caption*{Перевёрнутое Дерево-1}
    \end{minipage}
    \begin{minipage}{0.3 \textwidth}
        \centering
        \begin{tabular}{|c|c|}
            \hline
            $P$, Н & $y_{max}$, мм \\ \hline
            0.00 & 0.00 \\ \hline
            4.46 & 0.40 \\ \hline
            9.11 & 0.79 \\ \hline
            13.99 & 1.06  \\ \hline
            18.9 & 1.44 \\ \hline
            13.99 & 1.04 \\ \hline
            9.11 & 0.62 \\ \hline
            4.46 & 0.30 \\ \hline
            0.00 & 0.00 \\ \hline
        \end{tabular}
        \caption*{Дерево-2}
    \end{minipage}
    \begin{minipage}{0.3 \textwidth}
        \centering
        \begin{tabular}{|c|c|}
            \hline
            $P$, Н & $y_{max}$, мм \\ \hline
            0.00 & 0.00 \\ \hline
            4.46 & 0.41 \\ \hline
            9.11 & 0.82 \\ \hline
            13.99 & 1.25  \\ \hline
            18.9 & 1.68 \\ \hline
            13.99 & 1.28 \\ \hline
            9.11 & 0.95 \\ \hline
            4.46 & 0.44 \\ \hline
            0.00 & 0.05 \\ \hline
        \end{tabular}
        \caption*{Перевёрнутое Дерево-2}
    \end{minipage}
    \hfil
    \begin{minipage}{0.37 \textwidth}
        \centering
        \begin{tabular}{|c|c|}
            \hline
            $P$, Н & $y_{max}$, мм \\ \hline
            0.00 & 0.00 \\ \hline
            4.46 & 0.66 \\ \hline
            9.11 & 1.30 \\ \hline
            13.99 & 1.97  \\ \hline
            18.9 & 2.67 \\ \hline
            13.99 & 1.98 \\ \hline
            9.11 & 1.29 \\ \hline
            4.46 & 0.65 \\ \hline
            0.00 & 0.04 \\ \hline
        \end{tabular}
        \caption*{Сталь}
    \end{minipage}
    \begin{minipage}{0.23 \textwidth}
        \centering
        \begin{tabular}{|c|c|}
            \hline
            $P$, Н & $y_{max}$, мм \\ \hline
            0.00 & 0.00 \\ \hline
            4.46 & 0.59 \\ \hline
            9.11 & 1.25 \\ \hline
            13.99 & 1.93  \\ \hline
            18.9 & 2.66 \\ \hline
            13.99 & 1.97 \\ \hline
            9.11 & 1.28 \\ \hline
            4.46 & 0.64 \\ \hline
            0.00 & 0.03 \\ \hline
        \end{tabular}
        \caption*{Перевёрнутая Сталь}
    \end{minipage}
\end{table}

\vspace{4cm}

Модуль Юнга с помощью коэффициента наклона можно найти так:

\[E=\frac{kl^3}{4ab^3}\]

\clearpage

\begin{figure}[h!]
    \centering
    \includegraphics[width=1\textwidth]{дерево-1.png}
    \caption*{График для Дерево-1}
\end{figure}

$k = 6.78 \pm 0.09$ Н/мм

$E = (10.8 \pm 0.4)$ ГПа

\begin{figure}[h!]
    \centering
    \includegraphics[width=1\textwidth]{дерево-2.png}
    \caption*{График для Дерево-2}
\end{figure}

$k = 12.27 \pm 0.39$ Н/мм

$E = (19.6 \pm 2.1)$ ГПа

\clearpage

\begin{figure}[h!]
    \centering
    \includegraphics[width=1\textwidth]{сталь.png}
    \caption*{График для Сталь}
\end{figure}

$k = 7.14 \pm 0.05$ Н/мм

$E = (203.6 \pm 1.1)$ ГПа

\newpage

\section*{Определение модуля Юнга по измерениям растяжения проволоки}

\includegraphics[scale=0.5]{2.jpg}

\subsection*{Теоритические сведения}

\[F=\frac{ESr}{2lh}n\]

\subsection*{Результаты измерений и обработка данных}

\begin{table}[H]
    \centering
    \caption{Параметры прибора Лермантова}
    \begin{tabular}{|c|c|c|c|c|}
    \hline
    $d_{\text{пр}}$, мм & $r$, мм & $l$, см & $h$, см & $\sigma_{\text{пр}}$, Н/мм$^2$\\ \hline
    $0.46 \pm 0.01$            & $15 \pm 1$ & $177.6 \pm 0.5$  & $144.0\pm 0.5$   & 900\\ \hline
    \end{tabular}
\end{table}

Предельная нагрузка на разрушение: $P=45.9$ Н

\begin{table}[H]
    \centering
    \caption{Зависимость показаний шкалы от нагрузки}
    \begin{tabular}{|c|c|c|c|c|c|}
        \hline
        $P$, Н & $n_1 \downarrow$, см & $n_1\uparrow$, см & $n_2 \downarrow$, см & $n_2 \uparrow$, см \\ \hline
        4.9    & 13.5                 & 15.8              & 15.8                 & 15.2               \\ \hline
        7.3    & 16.1                 & 18.7              & 18.9                 & 18.4               \\ \hline
        9.7    & 18.6                 & 21.6              & 21.3                 & 21.3               \\ \hline
        12.1   & 21.8                 & 23.9              & 24.2                 & 24.1               \\ \hline
        14.5   & 24.4                 & 26.5              & 26.8                 & 26.8               \\ \hline
    \end{tabular}
\end{table}

\begin{figure}[h!]
    \centering
    \includegraphics[width=0.6\textwidth]{1.3.2 g.png}
    \caption*{График зависимости $n$ от $F$}
\end{figure}

Модуль Юнга с помощью коэффициента наклона можно найти так:

\[E=\frac{2lh}{kSr}\]

\[E=(178 \pm 18) \text{ГПа}\]

Исходя из табличных значений проволока сделана из стали.

\section*{Вывод}

В результате работы были исследованы упргуие деформации твердых тел. Подтверждена теоретическая модель, предсказывающая линейную зависимость. Получены значения модуля Юнга совпадают с табличными.

Табличные значения взяты с сайта \href{https://en.wikipedia.org/wiki/Young%27s_modulus}{Wikipedia}.

\end{document}