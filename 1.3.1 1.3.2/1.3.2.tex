\documentclass[a4paper, 12pt]{article}

% Русский язык
\usepackage[T2A]{fontenc}
\usepackage[utf8]{inputenc}
\usepackage[english,russian]{babel}

% Картинки
\usepackage{graphicx}
\graphicspath{{images/}}
\DeclareGraphicsExtensions{.pdf,.png,.jpg}
\usepackage{wrapfig}

% Математика
\usepackage{amsmath,amsfonts,amssymb,amsthm,mathtools} 

% Параметры страницы
\usepackage[left=2cm,right=2cm,top=2cm,bottom=3cm,bindingoffset=0cm]{geometry}

\usepackage{enumitem,xparse}
\usepackage{caption}
\usepackage{subcaption}
\usepackage{float}

\usepackage{hyperref}

\begin{document}

\newgeometry{left=2cm,right=2cm,top=2cm,bottom=1cm,bindingoffset=0cm}
\begin{titlepage}
    
    \begin{center}
        \vspace*{5cm}
        \Huge Московский Физико-Технический Институт
        \vspace*{2cm}\\
        \LARGE Отчет по эксперименту
        \\\vspace*{0.25cm}
        
        \noindent\rule{\textwidth}{1pt}
        \vspace*{-0.25cm}
        
        \huge \textbf{1.3.2 \\Определение модуля кручения стержня статическим методом}
        \noindent\rule{\textwidth}{1pt}


       \vfill
        \begin{flushright}
            \begin{minipage}{.4\textwidth}
            \Large Выполнил:\\ Студент 1 курса ФАКТ\\ Группа Б03-504 \\Подмосковнов Лев\\
            \end{minipage}
        \end{flushright}
        
        \vfill
        \normalsize Долгопрудный \\2025
        
    \end{center}
\end{titlepage}
\restoregeometry

\setcounter{page}{2}

\includegraphics[scale=0.5]{3.jpg}

\section*{Теоритические сведения}

\[f=\frac{\pi R^4G}{2l}\]

\newpage

\section*{Результаты измерений и обработка данных}

\begin{table}[h]
    \begin{center}
        \begin{tabular}{|c|c|c|c|c|}
            \hline 
            $m$, г & $\Delta x_2$, см & $\Delta x_3$, см\\ \hline
            0 & 0 & 0 & 0 \\ \hline
            150 & 2.9 & 2.7 & 2.8 \\ \hline
            300 & 5.6 & 5.8 & 5.6 \\ \hline
            450 & 8.9 & 8.5 & 8.4 \\ \hline
            600 & 11.7 & 11.1 & 11.3 \\ \hline
            750 & 14.4 & 14.1 & 14.0 \\ \hline
            900 & 16.9 & 17.4 & 17.8 \\ \hline
            750 & 14.6 & 14.5 & 14.6 \\ \hline
            600 & 11.6 & 11.6 & 11.6 \\ \hline
            450 & 8.6 & 8.7 & 8.6 \\ \hline
            300 & 5.8 & 5.8 & 5.9 \\ \hline
            150 & 2.9 & 2.8 & 2.8 \\ \hline
            0 & 0 & 0 & 0\\ \hline
        \end{tabular}
    \end{center}
\end{table}

\begin{figure}[h!]
    \centering
    \includegraphics[width=0.9\textwidth]{1.3.2.png}
    \caption*{Зависимость $\varphi (M)$}
\end{figure}

\clearpage

\[k = 0.367\] 
\[f = \frac{1}{f} = (2.725 \pm 0.127) \frac{\text{Н} \cdot \text{м}}{\text{рад}}\]
\[G=\frac{2l}{\pi R^4}f=(79.1 \pm 0.3)\text{ГПа}\]

\section*{Вывод}

В работе была исследована зависимость между кручением и нагрузкой. Был рассчитан модуль кручения для стального стержня. Полученное значение совпадает с табличным.

Табличные значения взяты с сайта \href{https://en.wikipedia.org/wiki/Shear_modulus}{Wikipedia}.

\end{document}