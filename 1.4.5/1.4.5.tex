\documentclass[a4paper, 12pt]{article}

% Русский язык
\usepackage[T2A]{fontenc}
\usepackage[utf8]{inputenc}
\usepackage[english,russian]{babel}

% Картинки
\usepackage{graphicx}
\graphicspath{{images/}}
\DeclareGraphicsExtensions{.pdf,.png,.jpg}
\usepackage{wrapfig}

% Математика
\usepackage{amsmath,amsfonts,amssymb,amsthm,mathtools} 

% Параметры страницы
\usepackage[left=2cm,right=2cm,top=2cm,bottom=3cm,bindingoffset=0cm]{geometry}

\usepackage{enumitem,xparse}
\usepackage{caption}
\usepackage{subcaption}
\usepackage{float}

\begin{document}

\newgeometry{left=2cm,right=2cm,top=2cm,bottom=1cm,bindingoffset=0cm}
\begin{titlepage}
    
    \begin{center}
        \vspace*{5cm}
        \Huge Московский Физико-Технический Институт
        \vspace*{2cm}\\
        \LARGE Отчет по эксперименту
        \\\vspace*{0.25cm}
        
        \noindent\rule{\textwidth}{1pt}
        \vspace*{-0.25cm}
        
        \huge \textbf{1.4.5 \\ Изучение колебаний струны}
        \noindent\rule{\textwidth}{1pt}

        \vspace{6cm}
        \vfill
        \begin{flushright}
            \begin{minipage}{.4\textwidth}
            \Large Выполнил:\\ Студент 1 курса ФАКТ\\ Группа Б03-504 \\Подмосковнов Лев\\
            \end{minipage}
        \end{flushright}
        
        \vfill
        \normalsize Долгопрудный \\2025
        
    \end{center}
\end{titlepage}
\restoregeometry

\setcounter{page}{2}

\section*{Теоритические сведения}

Формула для расчёта скорости распространения волны в среде

\[u=\sqrt{\frac{T}{\rho_l}}\]

Формула для расчёта частоты n-ой гармоники

\[\nu_n = \frac{n}{2L} \sqrt{\frac{T}{\rho_l}}\]

\section*{Ход работы}

\subsection*{Визуальное наблюдение стоячих волн}

Скорость распространения волн: $u=137$ м/с

\noindent Частота первой грамоники:$\nu_1=137$ Гц

\begin{table}[h]
    \centering
    \begin{tabular}{|c|c|c|c|c|}
        \hline
        $\nu_1$, \text{Гц} & $\nu_2$, \text{Гц} & $\nu_3$, \text{Гц} & $\nu_4$, \text{Гц} 4 & $\nu_5$, \text{Гц} \\
        \hline
        137 & 274 & 411 & 549 & 682 \\
        \hline
    \end{tabular}
    \caption*{Визуальное наблюдение стоячих волн}
\end{table}

\subsection*{Регистрация стоячих волн с помощью осциллографа}

\begin{table}[h]
    \begin{minipage}{0.2 \textwidth}
        \centering
        \begin{tabular}{|c|c|}
        \hline
        $n$ & $\nu_n$, \text{Гц} \\
        \hline
            1 & 136.99 \\ \hline
            2 & 263.16 \\ \hline
            3 & 434.78 \\ \hline
            4 & 555.56 \\ \hline
            5 & 689.66 \\ \hline
            6 & 869.57 \\ \hline
            7 & 1000 \\ \hline
            8 & 1111.11 \\ \hline
            9 & 1250 \\ \hline
            10 & 1428.57 \\ \hline
            \hline
        \end{tabular}
        \caption*{$m=1087,2$ г}
    \end{minipage}
    \begin{minipage}{0.19 \textwidth}
        \centering
        \begin{tabular}{|c|c|}
            \hline
            $n$ & $\nu_n$, \text{Гц} \\
            \hline
            1 & 166.67 \\ \hline
            2 & 336.40 \\ \hline
            3 & 504.70 \\ \hline
            4 & 674.00 \\ \hline
            5 & 843.80 \\ \hline
            6 & 1012.20 \\ \hline
            7 & 1183.50 \\ \hline
            8 & 1354.90 \\ \hline
            9 & 1526 \\ \hline
            10 & 1698.50 \\ \hline
            \hline
        \end{tabular}
        \caption*{$m=1574,6$ г}
    \end{minipage}
    \begin{minipage}{0.19 \textwidth}
        \centering
        \begin{tabular}{|c|c|}
            \hline
            $n$ & $\nu_n$, \text{Гц} \\
            \hline
            1 & 190 \\ \hline
            2 & 381 \\ \hline
            3 & 572.4 \\ \hline
            4 & 763 \\ \hline
            5 & 954.6 \\ \hline
            6 & 1146 \\ \hline
            7 & 1338.7 \\ \hline
            8 & 1533 \\ \hline
            9 & 1725 \\ \hline
            10 & 1920 \\ \hline
            \hline
        \end{tabular}
        \caption*{$m=2071,8$ г}
    \end{minipage}
    \begin{minipage}{0.19 \textwidth}
        \centering
        \begin{tabular}{|c|c|}
            \hline
            $n$ & $\nu_n$, \text{Гц} \\
            \hline
            1 & 209 \\ \hline
            2 & 419 \\ \hline
            3 & 629 \\ \hline
            4 & 839 \\ \hline
            5 & 1049 \\ \hline
            6 & 1259 \\ \hline
            7 & 1472 \\ \hline
            8 & 1683 \\ \hline
            9 & 1895 \\ \hline
            10 & 2108 \\ \hline
            \hline
        \end{tabular}
        \caption*{$m=2566,2$ г}
    \end{minipage}
    \begin{minipage}{0.19 \textwidth}
        \centering
        \begin{tabular}{|c|c|}
            \hline
            $n$ & $\nu_n$, \text{Гц} \\
            \hline
            1 & 229 \\ \hline
            2 & 458 \\ \hline
            3 & 688 \\ \hline
            4 & 917 \\ \hline
            5 & 1147 \\ \hline
            6 & 1377 \\ \hline
            7 & 1608 \\ \hline
            8 & 1840 \\ \hline
            9 & 2071 \\ \hline
            10 & 2303 \\ \hline
            \hline
        \end{tabular}
        \caption*{$m=3060,1$ г}
    \end{minipage}
\end{table}

\begin{figure}[H]
    \centering
    \includegraphics[width=0.6\textwidth]{1.png}
    \caption*{Фигура Лиссажу с одним самопересечением}
\end{figure}

Благодаря высокой добротности струны, возможно возбуждение её колебаний при кратных частотах генератора, меньших, чем $\nu_1$. Для наблюдения явления переключим осциллограф в режим (X-Y) и настроим установку на наблюдение основной гармоники. Затем уменьшим частоту возбуждения в два раза, установивна генераторе $\nu=\frac{1}{2}\nu_1$. На экране осциллографа наблюдается фигура Лиссажу с одним самопересечением

\section*{Обработка результатов}

\begin{figure}[H]
    \centering
    \includegraphics[width=0.75\textwidth]{2.png}
    \caption*{Зависимость $\nu_n$ и их аппроксимация линейной функцией по МНК.}
\end{figure}

\begin{figure}[H]
    \centering
    \includegraphics[width=1\textwidth]{3.png}
    \caption*{Зависимость $u^2$ от $T$}
\end{figure}

Из аппроксимации прямой получаем погонную плотность: $\rho_l=(582 \pm 4)$ мг/м. 

\newpage

\section*{Вывод}

В работе были изучены поперечные стоячие волны на тонкой натянутой струне, были измерены собственные частоты её колебаний, измерена скорость распространения волн в струне и линейная плотность струны. Экспериментальные графики зависимостей $\nu_n(n)$ и $u^2(T)$ хорошо ложатся на аппроксимирующие прямые, но эти прямые не проходят через начало координат. Однако отклонение аппроксимирующих прямых от начала координат по оси ординат мало ($\sim 1\%$) по сравнению с значениями ординат экспериментальных точек. Отличие измеренного значения линейной плотности струны от указанного на установке составляет $6 \%$. Само значение $\rho$ измерено с достаточно высокой точностью $\varepsilon_{\rho} = 7 \cdot 10^{-3} \ \%$. Отличие $\rho$ от указанного на установке значения более, чем на погрешность, может быть связано с:
\begin{itemize}
\item[1) ] Неточностью определения собственных частот $\nu_n$ из-за возникновения нелинейных эффектов при резонансе, и, как следствие, неточностью в определении скорости распространения $u$ волны в струне.
\item[2) ] Неучтением погрешностей измерения собственных частот.
\item[3) ] Недостаточным количеством экспериментальных точек на графике $u^2(T)$, то есть недостаточным количеством опытов по измерению собственных частот струны в зависимости от силы натяжения нити $T$.
\end{itemize}

\end{document}