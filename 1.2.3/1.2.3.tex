\documentclass[a4paper, 12pt]{article}

% Русский язык
\usepackage[T2A]{fontenc}
\usepackage[utf8]{inputenc}
\usepackage[english,russian]{babel}

% Картинки
\usepackage{graphicx}
\graphicspath{{images/}}
\DeclareGraphicsExtensions{.pdf,.png,.jpg}
\usepackage{wrapfig}

% Математика
\usepackage{amsmath,amsfonts,amssymb,amsthm,mathtools} 

% Параметры страницы
\usepackage[left=2cm,right=2cm,top=2cm,bottom=3cm,bindingoffset=0cm]{geometry}

\usepackage{enumitem,xparse}
\usepackage{caption}
\usepackage{subcaption}
\usepackage{float}

\usepackage{hyperref}

\begin{document}

\newgeometry{left=2cm,right=2cm,top=2cm,bottom=1cm,bindingoffset=0cm}

\begin{titlepage}
\begin{center}

    \vspace*{5cm}

    {\Huge Московский Физико-Технический Институт}\\[2cm]

    {\LARGE Отчёт по эксперименту}\\[0.25cm]

    \noindent\rule{\textwidth}{1pt}
    \vspace*{-0.25cm}

    {\huge \textbf{1.2.3 \\ Определение моментов инерции твёрдых тел\\ с помощью трифилярного подвеса}}

    \noindent\rule{\textwidth}{1pt}

    \vfill

\end{center}

\begin{flushright}
    \begin{minipage}{0.45\textwidth}
        {\Large
        Выполнил:\\
        Студент 1 курса ФАКТ\\
        Группа Б03-504\\
        Подмосковнов Лев
        }
    \end{minipage}
\end{flushright}

\vfill

\begin{center}
    Долгопрудный\\2025
\end{center}

\end{titlepage}
\restoregeometry

\setcounter{page}{2}

\section*{Теоритические сведения}

\[I=\int r^2 dm\]
Для цилиндра: $I=\frac{mr^2}{2}$

\noindentДля кольца: $I=\frac{m{R_1^2+R_2^2}}{2}$

\[I=kmT^2\]

\[k=\frac{gRr}{4\pi^2z_0}\]

\section*{Результаты измерений и обработка данных}

\[R=(114.6 \pm 0.5)\text{мм}\]
\[r=(30.5 \pm 0.3)\text{мм}\]
\[m=(934.7 \pm 0.5)\text{г}\]
\[z_0=(218.1  \pm 0.5)\text{см}\]
\[k=(4.04 \cdot 10^{-4})\; \text{м}^2 / \text{c}^2\]

\subsection*{Пустая платформа}

\begin{table}[h]
    \begin{center}
        \begin{tabular}{|c|c|c|c|}
            \hline 
            № опыта & Кол-во колебаний $n$ & $t$, c & $I,\; 10^{-3} \text{кг} \cdot \text{м}^2$\\ \hline
            1 & 20 & 87.84 & 7.28 \\ \hline
            2 & 20 & 88.01 & 7.31 \\ \hline
            3 & 20 & 88.01 & 7.31 \\ \hline
        \end{tabular}
    \end{center}
\end{table}

Радиус платформы: $R_0=(12.5 \pm 0.5)\text{см}$

Теоритический момент инерции: $I = (7.30 \cdot 10^{-3}) \text{кг} \cdot \text{м}^2$ \

\subsection*{Проверка аддитивности моментов инерции}

Радиус диска: $R_0=8$ см

\noindentМасса диска: $m=881.7$ г

\noindentВнутренний радиус кольца: $R_1=8$ см

\noindentВнешний радиус кольца: $R_2=8.4$ см

\noindentМасса кольца: $m=772.1$ г

\clearpage

\begin{table}[h]
    \begin{center}
        \begin{tabular}{|c|c|c|c|}
            \hline 
            № опыта & Кол-во колебаний $n$ & $t$, c & $I,\; 10^{-3} \text{кг} \cdot \text{м}^2$\\ \hline
            1 & 20 & 74.02 & 3.51 \\ \hline
            2 & 20 & 74.21 & 3.55 \\ \hline
            3 & 20 & 73.97 & 3.51 \\ \hline
            4 & 20 & 74.28 & 3.57 \\ \hline
            5 & 20 & 74.14 & 3.53 \\ \hline
        \end{tabular}
    \end{center}
    \caption*{Диск}
\end{table}

Теоритический момент инерции: $I = (3.53 \cdot 10^{-3}) \text{кг} \cdot \text{м}^2$ \

\begin{table}[h]
    \begin{center}
        \begin{tabular}{|c|c|c|c|}
            \hline 
            № опыта & Кол-во колебаний $n$ & $t$, c & $I,\; 10^{-3} \text{кг} \cdot \text{м}^2$\\ \hline
            1 & 20 & 83.63 & 4.48 \\ \hline
            2 & 20 & 83.44 & 4.42 \\ \hline
            3 & 20 & 83.66 & 4.56 \\ \hline
            4 & 20 & 83.63 & 4.48 \\ \hline
            5 & 20 & 83.58 & 4.33 \\ \hline
        \end{tabular}
    \end{center}
    \caption*{Кольцо}
\end{table}

Теоритический момент инерции: $I = (4.48 \cdot 10^{-3}) \text{кг} \cdot \text{м}^2$ \

\begin{table}[h]
    \begin{center}
        \begin{tabular}{|c|c|c|c|}
            \hline 
            № опыта & Кол-во колебаний $n$ & $t$, c & $I,\; 10^{-3} \text{кг} \cdot \text{м}^2$\\ \hline
            1 & 20 & 75.29 & 8.03 \\ \hline
            2 & 20 & 75.28 & 8.03 \\ \hline
            3 & 20 & 75.20 & 7.91 \\ \hline
            4 & 20 & 75.35 & 8.06 \\ \hline
            5 & 20 & 75.31 & 8.04 \\ \hline
        \end{tabular}
    \end{center}
    \caption*{Кольцо+Диск}
\end{table}

Теоритический момент инерции: $I = (8.01 \cdot 10^{-3}) \text{кг} \cdot \text{м}^2$ \

Как видно из таблицы, аддитивность моментов инерции соблюдается.

\subsection*{Проверка теоремы Гюйгенса-Штейнера}

\begin{table}[h]
    \centering
    \begin{tabular}{|c|c|c|c|}
        \hline
        № & $h$, мм & $T$, c & $I, 10^{-3} \text{ кг}\cdot\text{м}^2$ \\ \hline
        1  & 0  & 3.094 & 1.58 \\ \hline
        2  & 5  & 3.098 & 1.61 \\ \hline
        3  & 10 & 3.116 & 1.71 \\ \hline
        4  & 15 & 3.142 & 1.86 \\ \hline
        5  & 20 & 3.188 & 2.12 \\ \hline
        6  & 25 & 3.226 & 2.34 \\ \hline
        7  & 30 & 3.294 & 2.75 \\ \hline
        8  & 35 & 3.370 & 3.21 \\ \hline
        9  & 40 & 3.444 & 3.66 \\ \hline
        10 & 45 & 3.550 & 4.34 \\ \hline
        11 & 50 & 3.634 & 4.88 \\ \hline
    \end{tabular}
\end{table}

\begin{figure}[h!]
    \centering
    \includegraphics[width=1\textwidth]{1.png}
\end{figure}

Найдем коэффициенты по МНК:
		
		\[I = b = (1.142 \pm 0.005) \cdot 10^{-3} \text{ кг}\cdot\text{м}^2\]
		\[m = k = (1.336 \pm 0.006) \text{ кг}\]
		
		Как видно из эксперимента, формула Гюйгенса-Штейнера работает, а массы цилиндра, вычисленные по МНК, лежат в пределах допустимой погрешности.

\clearpage

\section*{Вывод}
С помощью трифилярного подвеса можно определять момент инерции с достаточно большой точностью. Такая точность обусловлена малой погрешностью измерения времени и условиями, при которых колебания подвеса можно считать слабозатухающими.

Мы экспериментально доказали аддитивность моментов инерции с помощью различных тел.

Полученная зависимость $I(h^2)$ аппроксимируется линейной зависимостью, что подвтерждает формулу Гюйгенса-Штейнера ($I = I_c + Mh^2$, где $I$ -- момент инерции тела, $I_c$ --момент инерции тела относительно центра, $M$ -- масса тела, а $h$ -- расстояние между двумя осями, в нашем случае -- между осью вращения и половинками диска).
		

\end{document}