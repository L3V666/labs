\documentclass[a4paper, 12pt]{article}

% Русский язык
\usepackage[T2A]{fontenc}
\usepackage[utf8]{inputenc}
\usepackage[english,russian]{babel}

% Картинки
\usepackage{graphicx}
\graphicspath{{images/}}
\DeclareGraphicsExtensions{.pdf,.png,.jpg}
\usepackage{wrapfig}

% Математика
\usepackage{amsmath,amsfonts,amssymb,amsthm,mathtools} 

% Параметры страницы
\usepackage[left=2cm,right=2cm,top=2cm,bottom=3cm,bindingoffset=0cm]{geometry}

\usepackage{enumitem,xparse}
\usepackage{caption}
\usepackage{subcaption}
\usepackage{float}

\begin{document}

\newgeometry{left=2cm,right=2cm,top=2cm,bottom=1cm,bindingoffset=0cm}
\begin{titlepage}
    
    \begin{center}
        \vspace*{5cm}
        \Huge Московский Физико-Технический Институт
        \vspace*{2cm}\\
        \LARGE Отчет по эксперименту
        \\\vspace*{0.25cm}
        
        \noindent\rule{\textwidth}{1pt}
        \vspace*{-0.25cm}
        
        \huge \textbf{1.2.5 \\ Определение скорости полёта пули при помощи баллистического маятника}
        \noindent\rule{\textwidth}{1pt}

        \vspace{6cm}
        \vfill
        \begin{flushright}
            \begin{minipage}{.4\textwidth}
            \Large Выполнил:\\ Студент 1 курса ФАКТ\\ Группа Б03-504 \\Подмосковнов Лев\\
            \end{minipage}
        \end{flushright}
        
        \vfill
        \normalsize Долгопрудный \\2025
        
    \end{center}
\end{titlepage}
\restoregeometry

\setcounter{page}{2}

\section*{Теоретичиские сведения}

\section*{Ход работы}


\begin{table}[!h]
\begin{center}
\begin{tabular}{|c|c|}
\hline
№ пули & m, г \\ \hline
1	& 0.512 \\ \hline
2	& 0.507 \\ \hline
3	& 0.506 \\ \hline
4	& 0.509 \\ \hline
5	& 0.504 \\ \hline
6	& 0.502 \\ \hline
7	& 0.504 \\ \hline
8	& 0.510 \\ \hline
9	& 0.511 \\ \hline
10	& 0.504 \\ \hline
\end{tabular}
\end{center}
\end{table}

\subsection*{Метод баллистического маятника, совершающего поступательное движение}

Расстояние $L=222\pm0.5$ см

\noindent Масса цилиндра $M=2925\pm5$ г

\begin{table}[!h]
\begin{center}
\begin{tabular}{|c|c|c|}
\hline
№ пули & $\Delta x, $ мм & $u, $ м/с \\ \hline
1	& 9.3 & 111.23 \\ \hline
2	& 9.8 & 118.37 \\ \hline
3	& 9.5 & 114.97 \\ \hline
4	& 9.5 & 114.29 \\ \hline
5	& 9.6 & 116.64 \\ \hline
\end{tabular}
\end{center}
\end{table}

Средняя скорость $\overline{u}=115.1$ м/с. Погрешность $\varepsilon_u=1$ \%

\subsection*{Вывод}

В данной работе мы экспериментально определили скорость полета пули, как видно с высокой точностью. Отличия скоростей у разных пуль возникает из за во-первых их отличия по массе, а во вторых из-за параметров выстрела (аэродинамика и начальное положение в винтовке)

\newpage

\subsection*{Метод крутильного баллистического маятника}

$r = 20,5\pm0.5$ см

\noindent$R = 33,5\pm0.5$ см

\noindent$d = 141\pm0.5$ см

\begin{table}[!h]
\begin{center}
\begin{tabular}{|c|c|}
\hline
№ пули & $T_1$, с\\ \hline
1	& 6.66 \\ \hline
2	& 6.71 \\ \hline
3	& 6.70 \\ \hline
4	& 6.69 \\ \hline
5	& 6.67 \\ \hline
\end{tabular}
\end{center}
\end{table}

\begin{table}[!h]
\begin{center}
\begin{tabular}{|c|c|}
\hline
№ измерения & $T_2$, с\\ \hline
1	& 4.72 \\ \hline
2	& 4.68 \\ \hline
3	& 4.70 \\ \hline
4	& 4.69 \\ \hline
5	& 4.67 \\ \hline
\end{tabular}
\end{center}
\end{table}

Среднее значение: $\langle T_1 \rangle = 6.69$ с

Среднее значение: $\langle T_2 \rangle = 4.69$ с

Величина $\sqrt{kI} = (0.609 \pm 0.006) \tfrac{\text{кг}\cdot\text{м}^2}{c}$

\begin{table}[!h]
\begin{center}
\begin{tabular}{|c|c|c|}
\hline
№ пули & $\Delta x, $ мм & $u, $ м/с \\ \hline
6	& 4 & 167.88 \\ \hline
7	& 4.5 & 188.12\\ \hline
8	& 4.2 & 173.51 \\ \hline
9	& 4.4 & 181.42 \\ \hline
10	& 4.4 & 183.94 \\ \hline
\end{tabular}
\end{center}
\end{table}

\section*{Вывод}

В данной работе мы экспериментально посчитали скорость полета пули. Отличия скоростей возникают из-за разных начальных прамаетров и аэродинамики.

\end{document}